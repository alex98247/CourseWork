\documentclass[a4paper,article,14pt]{extarticle}

% Глобальные поля
\usepackage[left=30mm, top=20mm, right=15mm, bottom=20mm,nohead, includefoot,footskip=35pt]{geometry}

% Язык, кодировка, шрифт
\usepackage[utf8]{inputenc}
\usepackage[T2A]{fontenc}
\usepackage[english,russian]{babel}

\renewcommand{\baselinestretch}{1.5}

\usepackage{tikz}
\usetikzlibrary{positioning}
\usepackage{graphicx}
\usepackage{float}
\usepackage{amsmath}
\usepackage{amssymb}

\usepackage{titlesec}
\titleformat{\section}[block]{\Large\bfseries\filcenter}{}{1em}{}

\usepackage{indentfirst}

\begin{document}
\section*{Текст выступления}
Уважаемый председатель! Уважаемые члены государственной аттестационной комиссии!

Вашему вниманию представляется выпускная квалификационная работа  на тему: "\textbf{\textit{Оценка области захвата для систем ФАПЧ 3 порядка}}". 

Система фазовой автоподстройки частоты (ФАПЧ)~--- система с обратной связью, подстраивающая частоту сигнала генератора, управляемого напряжением (ГУН) под частоту опорного сигнала. В настоящее время системы ФАПЧ применяются в телекоммуникационном оборудовании, навигационных системах и других областях. Для физической реализации таких систем инженерам необходимо определять полосу захвата, которая определяется областью параметров, обеспечивающей глобальную устойчивость системы. 

Моей задачей было исследование систем фазовой автоподстройки частоты третьего порядка с передаточными функциями фильтров трех видов с целью получения оценок полос захвата.

Опираясь на частотную теорему Леонова Г. А., мною были получены оценки полосы захвата систем ФАПЧ третьего порядка с передаточными функциями фильтров трех видов. Для систем ФАПЧ с фильтром, определяемым передаточной функцией первого вида, была получена аналитическая оценка полосы захвата. Так же был представлен график, который иллюстрирует зависимость $\nu^2$ от $\tau_{p1}$, $\tau_{p2}$. Для систем ФАПЧ, с фильтром определяемым передаточной функцией второго вида, была получена аналитическая оценка полосы захвата. Так же был представлен график, который демонстрирует, что аналитическую оценку можно улучшить. Для систем ФАПЧ с фильтром, определяемым передаточной функцией третьего вида исследовалась ранее Леоновым~Г.\:А. и Кузнецовым~Н.\:В. Для нее был восстановлен вывод.

Во время изучения моей выпускной квалификационной работы у рецензента возник вопрос: каким образом были верифицированы полученные оценки? Отвечая на этот вопрос хочется отметить, что для первой и второй передаточной функций проводилось численное моделирование на 10 000 точках, при этом для первой передаточной функции максимальная разность по модулю между аналитической  и численной оценками составила 7.9533e-03. Для второй передаточной функции максимальная разность по модулю между аналитической оценкой и промоделированной численной оценкой составила 3.3111e-07. Для третьей передаточной функции оценка, полученная в выпускной квалификационной работе совпала с оценкой полученной Леоновым~Г.\:А. и Кузнецовым~Н.\:В.

\end{document}