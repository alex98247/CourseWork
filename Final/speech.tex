\documentclass[a4paper,article,14pt]{extarticle}

% Язык, кодировка, шрифт
\usepackage[utf8]{inputenc}
\usepackage[T2A]{fontenc}
\usepackage[english,russian]{babel}

\usepackage{tikz}
\usetikzlibrary{positioning}
\usepackage{graphicx}
\usepackage{float}
\usepackage{amsmath}
\usepackage{amssymb}

\usepackage{titlesec}
\titleformat{\section}[block]{\Large\bfseries\filcenter}{}{1em}{}
\titleformat{\subsection}[block]{\bfseries\filcenter}{}{1em}{}

\usepackage{indentfirst}

\begin{document}
\section*{Текст выступления}
\subsection*{Слайд 1}
  Добрый день, уважаемые коллеги. Меня зовут Миронов Алексей. Сегодня я буду рассказывать про оценку полосы захвата для систем фазовой автоподстройки частоты третьего порядка. 

\subsection*{Слайд 2}
  Для начала разберем что такое система фазовой автоподстройки частоты и принцип ее работы. Система фазовой автоподстройки частоты (ФАПЧ)~--- система с обратной связью, подстраивающая частоту сигнала генератора, управляемого напряжением (ГУН) под частоту опорного сигнала.
  
На слайде представлена схема ФАПЧ. Опорный сигнал вместе с сигналом с генератора, управляемого напряжением (ГУН) поступает в фазовый детектор. Фазовый детектор сравнивает частоты сигналов и на выходе получается сигнал следующего вида: $sin(\omega_e t)$. Здесь синус называется характеристикой фазового детектора. Далее выходной сигнал фазового детектора поступает на вход фильтра нижних частот, который является линейным блоком и может быть описан передаточной функцией $W(s)$. Выход фильтра поступает на вход генератора, управляемого напряжением, и происходит подстройка частоты ГУН. 
  
  Системы фазовой автоподстройки частоты широко применяются в различных системах, таких как телекоммуникационное оборудование, навигационное оборудование (GPS, ГЛОНАСС, Галилео). Также системы ФАПЧ применяются в компьютерах. Системы фазовой автоподстройки частоты нашли широкое применение и в военной индустрии. Например, военными был разработан метод попеременной смены частоты сигнала для того, чтобы сигнал было сложнее заглушить. Для реализации такого метода также применяется система фазовой автоподстройки частоты.
    
    \subsection*{Слайд 3}
На слайде представлена система дифференциальных уравнений описывающих ФАПЧ. $A$~--- постоянная матрица $n \times n$, $B$ и $C$ постоянные $n$~--- мерные векторы, $D$~--- константа, $x(t)$~--- $n$-мерный вектор состояний системы, $K_{vco}$~--- коэффициент передачи. Параметр $\gamma$ определяется следующим образом. Где $\omega^{free}_{e}=\omega_{ref}-\omega_{vco}^{free}$~--- разность частоты опорного сигнала и частоты свободных колебаний ГУН.

\subsection*{Слайд 4}
    Введём определение полосы захвата. Полоса захвата~--- максимальная разность по модулю частот опорного сигнала и ГУН $|\omega_p|$ такая, что система дифференциальных уравнений ФАПЧ глобально асимптотически устойчива для всех $0 \leqslant |\omega_e^{free}|<|\omega_p|$. 
    
Задача: оценить полосы захвата для передаточных функций фильтров, представленных на слайде.
    
\subsection*{Слайд 5}
В 1992 году Ленов~Г.\:А., Райтман~Ф., Смирнова~В.\:Б. доказали теорему, приведенную на слайде. Пусть пара $(A, B)$ вполне управляема, все собственные значения матрицы $A$ имеют отрицательные вещественные части и существуют числа $\varepsilon > 0$, $\delta > 0$, $\tau \geqslant 0$, и $\varkappa$, такие что имеют место неравенства. Тогда система, представленная на слайде глобально асимптотически устойчива.

\subsection*{Слайд 6}
Для получения оценок полосы захвата необходимо выбрать $\varepsilon, \delta, \varkappa, \tau$, удовлетворяющие первому условию теоремы так, чтобы максимизировать $\nu$.  Из соотношений, приведенных на слайде, получим максимальный $\omega_e^{free}$, при котором система глобально асимптотически устойчива.

\subsection*{Слайд 7}
Для первой передаточной функции была найдена оценка полосы захвата. На слайде представлены два графика: слева~--- численная оценка $\nu^2$, полученная в MATLAB при помощи функции fmincon, справа~--- график функции $\nu^2$ построенной по аналитической оценки.

Для второй передаточной функции была найдена оценка полосы захвата. На слайде представлен график зависимости $\nu^2$ от $\tau_{p1}$ и $\tau_{z1}$. Красным цветом представлена численная оценка $\nu^2$ согласно 1 и 2 условиям теоремы в MATLAB с помощью функции fmincon. Синим цветом представлен график $\nu^2$, полученный из второго условия теоремы при справедливости первого условия теоремы. Из графика видно, что аналитическую оценку можно улучшить поскольку график аналитической оценки $\nu^2$ расположен ниже численной.

Во время изучения моей выпускной квалификацилнной работы у рецензента возник вопрос относительно того как валидируются, полученные оценки? Мною были учтены примечания рецензента. На данном слайде представлены ответы на поступившие мне вопросы. 

\subsection*{Слайд 8}
Рассмотрим третью передаточную функцию. Система ФАПЧ с фильтром описываемым такой передаточной функцией исследовалась Леоновым~Г.\:А. и Кузнецовым~Н.\:В. Для нее был восстановлен вывод оценки после захвата.

\end{document}