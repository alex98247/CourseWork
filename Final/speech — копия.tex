\documentclass[a4paper,article,14pt]{extarticle}

% Язык, кодировка, шрифт
\usepackage[utf8]{inputenc}
\usepackage[T2A]{fontenc}
\usepackage[english,russian]{babel}

\usepackage{tikz}
\usetikzlibrary{positioning}
\usepackage{graphicx}
\usepackage{float}
\usepackage{amsmath}
\usepackage{amssymb}

\usepackage{titlesec}
\titleformat{\section}[block]{\Large\bfseries\filcenter}{}{1em}{}

\newtheorem{theorem}{Теорема}

\begin{document}
В рамках выпускной квалификационной работы я занимался исследованием устойчивости систем фазовой автоподстройки частоты третьего порядка. Эта тема является актуальной, что подтверждается большим количеством статей в международных высокорейтинговых журналах по электротехнике и системам связи с высоким индексом цитирований. При исследовании систем ФАПЧ третьего и более высоких порядков основной акцент делается на  численных результатах. В выпускной квалификационной работе получены аналитические оценки полосы захвата, которые были подтверждены численно.

\begin{table}[H]
\begin{center}
\begin{tabular}{{|p{0.33\textwidth}|p{0.6\textwidth}|}}
\hline
Дата & Этап \\
\hline
15.09.2019~-- 27.11.2019 & Изучение системы фазовой автоподстройки частоты \\
 & Консультации с научным руководителем \\
\hline
27.11.2019~-- 30.01.2020 & Получение оценки полосы захвата систем ФАПЧ с первой передаточной функцией \\ & Моделирование результатов в MATLAB \\ & Консультации с научным руководителем \\
\hline
30.01.2020~-- 01.03.2020 & Получение оценки полосы захвата систем ФАПЧ со второй передаточной функцией \\ & Моделирование результатов в MATLAB \\ & Консультации с научным руководителем \\
\hline
01.03.2020~-- 02.04.2020 & Восстановление вывода оценки полосы захвата систем ФАПЧ с третьей передаточной функцией \\ & Консультации с научным руководителем \\
\hline
\end{tabular}
\end{center}
\end{table} 

\section*{Заключение}
В рамках ВКР работы были рассмотрены различные передаточные функции фильтров. Для них были получены аналитические оценки полосы захвата. Так же был восстановлен вывод оценки полосы захвата, для фильтра с передаточной функцией, которая исследовалась ранее Леоновым~Г.\:А.
 
\pagebreak
\section*{Приложения}
\begin{itemize}
	\item Презентация
	\item Текст выступления
\end{itemize}

\pagebreak
\section*{Текст выступления}
  Добрый день, уважаемые коллеги. Меня зовут Миронов Алексей. Сегодня я буду рассказывать про оценку полосы захвата для систем фазовой автоподстройки частоты третьего порядка. 

  Для начала разберем что такое система фазовой автоподстройки частоты и принцип ее работы. Система фазовой автоподстройки частоты (ФАПЧ)~--- система с обратной связью, подстраивающая частоту сигнала генератора, управляемого напряжением (ГУН) под частоту опорного сигнала.
  
На слайде представлена схема ФАПЧ. Опорный сигнал вместе с сигналом с генератора, управляемого напряжением (ГУН) поступает в фазовый детектор. Фазовый детектор сравнивает частоты сигналов и на выходе получается сигнал следующего вида: $sin(\omega_e t)$. Здесь синус называется характеристикой фазового детектора. Далее выходной сигнал фазового детектора поступает на вход фильтра нижних частот, который является линейным блоком и может быть описан передаточной функцией $W(s)$. Выход фильтра поступает на вход генератора, управляемого напряжением, и происходит подстройка частоты ГУН. 
  
  Системы фазовой автоподстройки частоты широко применяются в различных системах, таких как телекоммуникационное оборудование, навигационное оборудование (GPS, ГЛОНАСС, Галилео). Также системы ФАПЧ применяются в компьютерах. Системы фазовой автоподстройки частоты нашли широкое применение и в военной индустрии. Например, военными был разработан метод попеременной смены частоты сигнала для того, чтобы сигнал было сложнее заглушить. Для реализации такого метода также применяется система фазовой автоподстройки частоты.
    
На слайде представлена система дифференциальных уравнений описывающих ФАПЧ. $A$~--- постоянная матрица $n \times n$, $B$ и $C$ постоянные $n$~--- мерные векторы, $D$~--- константа, $x(t)$~--- $n$-мерный вектор состояний системы, $K_{vco}$~--- коэффициент передачи. Параметр $\gamma$ определяется следующим образом. Где $\omega^{free}_{e}=\omega_{ref}-\omega_{vco}^{free}$~--- разность частоты опорного сигнала и частоты свободных колебаний ГУН.
    
    Введём определение полосы захвата. Полоса захвата~--- максимальная разность по модулю частот опорного сигнала и ГУН $|\omega_p|$ такая, что система дифференциальных уравнений ФАПЧ глобально асимптотически устойчива для всех $0 \leqslant |\omega_e^{free}|<|\omega_p|$. 
    
Задача: оценить полосы захвата для передаточных функций фильтров, представленных на слайде.
    
В 1992 году Ленов~Г.\:А., Райтман~Ф., Смирнова~В.\:Б. доказали теорему, приведенную на слайде. Пусть пара $(A, B)$ вполне управляема, все собственные значения матрицы $A$ имеют отрицательные вещественные части и существуют числа $\varepsilon > 0$, $\delta > 0$, $\tau \geqslant 0$, и $\varkappa$, такие что имеют место неравенства. Тогда система, представленная на слайде глобально асимптотически устойчива.

Рассмотрим первую передаточную функцию. Положим $\tau = 0$ и $\varkappa = 1$. Подставим передаточную функцию в первое условие теоремы и перенесем все в левую часть неравенства. Для максимизации оценки $\nu$ положим значения параметров $\varkappa, \varepsilon, \tau, \delta$ как представлено на слайде. Тогда, подставив значения параметров в первое условие теоремы и применив второе, получим получим наибольшее значение параметра $\nu$, при котором система глобально асимптотически устойчива. Из соотношения на $\nu$ и $\gamma$, а также из соотношения на $\gamma$ и $\omega_e^{free}$ получим оценку полосы захвата.

На слайде представлены два графика: слева~--- численная оценка $\nu^2$, полученная в MATLAB при помощи функции fmincon, справа~--- график функции $\nu^2$ построенной по аналитической оценки.

Рассмотрим вторую передаточную функцию с ограничениями, представленными на слайде. Положим $\varkappa=1$, $\tau=0$. Оценка $\nu$ будет наибольшей, если параметры $\varepsilon$, $\delta$ лежат на одной из граней выпуклого многоугольника, ограниченного прямыми $\delta = 0$, $\varepsilon = 0$ и соотношениями представленными на слайде. где $z = \frac{\tau_{p1}}{\tau_{z1}}$, $q = 2z - \frac{1}{2} - \frac{1}{2}z^2$. 

Рассмотрим график зависимости $\nu^2$ от $\tau_{p1}$ и $\tau_{z1}$. Красным цветом представлена численная оценка $\nu^2$ согласно 1 и 2 условиям теоремы в MATLAB с помощью функции fmincon. Синим цветом представлен график $\nu^2$, полученный из второго условия теоремы при справедливости первого условия теоремы. Из графика видно, что аналитическую оценку можно улучшить поскольку график аналитической оценки $\nu^2$ расположен ниже численной.

Рассмотрим третью передаточную функцию. Положим $\tau = 0$ и $\varkappa = 1$. Подставим передаточную функцию в первое условие теоремы и перенесем все в левую часть неравенства. Для максимизации оценки $\nu$ положим значения параметров $\varkappa, \varepsilon, \tau, \delta$ как представлено на слайде. При этом в условии теоремы $\delta$ и $\varepsilon$ должны быть строго положительны. Для этого достаточно потребовать положительность числителя $\delta$ это также будет гарантировать и положительность знаменателя $\delta$, при этом $\varepsilon$ будет также положительно, поэтому условия теоремы выполняются. Тогда, подставив значения параметров в первое условие теоремы и применив второе, получим оценку, представленную на слайде.

Подведем итог, того, что было проделано в данной работе. Для первого фильтра была найдена оценка полосы захвата. Для второго фильтра была также найдена оценка полосы захвата, которую можно улучшить. Для третьего фильтра, который исследовался Леоновым~Г.\:А. и Кузнецовым~Н.\:В. был восстановлен вывод оценки после захвата.

\end{document}