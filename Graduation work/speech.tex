\documentclass[a4paper,article,14pt]{extarticle}

% Язык, кодировка, шрифт
\usepackage[utf8]{inputenc}
\usepackage[T2A]{fontenc}
\usepackage[english,russian]{babel}

\usepackage{tikz}
\usetikzlibrary{positioning}
\usepackage{graphicx}
\usepackage{float}
\usepackage{amsmath}
\usepackage{amssymb}

\newtheorem{theorem}{Теорема}

\begin{document}
  Добрый день, уважаемые коллеги. Меня зовут Миронов Алексей. Сегодня я буду рассказывать про оценку полосы захвата для систем фазовой автоподстройки частоты третьего порядка. 

  Для начала разберем что такое система фазовой автоподстройки частоты и принцип ее работы. Система фазовой автоподстройки частоты (ФАПЧ)~--- система с обратной связью, подстраивающая частоту сигнала генератора, управляемого напряжением (ГУН) под частоту опорного сигнала.
  
   На рис. \ref{PLL-img} приведена классическая схема кольца фазовой автоподстройки частоты. 
\begin{figure}[H]
\begin{center}
\resizebox{\textwidth}{!}{%
\begin{tikzpicture}
\node at (0.5,0) [draw, text width=3cm, align=center, rectangle] (a) {Фазовый детектор \\ (компаратор)};
\node at (5.2,0) [draw, label=below:{Фильтр}, align=center, rectangle] (loop_filter) {$W(s)$};
\node at (10.5,0) [draw, label={[align=center]below:Генератор,\\управляемый напряжением}] (VCO) {$\omega_{vco}(t) = \omega_{vco}^{free}+K_{vco}\upsilon_f(t)$};

\draw[->] (-3.3,0) -- (a.west) node[midway,above,inner sep=2pt] {sin($\omega_{ref}t$)};
\draw[->] (a.east) -- (loop_filter.west) node[midway,above,inner sep=1pt] {$\upsilon_e(\omega_{e}t)$};
\draw[->] (loop_filter.east) -- (VCO.west) node[midway,above,inner sep=2pt] {$\upsilon_f(t)$};
\draw[->] (VCO.east) -- (17,0) node[midway,above,inner sep=2pt] {cos($\omega_{vco}t$)};
\draw[->] (15,0) --  +(0,-2) --  +(-14.5,-2) --  (a.south);

\end{tikzpicture}
}
\end{center}
\caption{Схема классической системы ФАПЧ, где $\omega_{ref}$~--- частота опорного сигнала, $\omega_{vco}$~--- частота сигнала ГУН, $\upsilon_f(t)$~--- выходной сигнал фильтра, $\omega_{vco}^{free}$~--- частота свободных колебаний ГУН, $\upsilon_e(\omega_{e}t)$~-- характеристика фазового детектора, $\omega_e = \omega_{ref}-\omega_{vco}$ \label{PLL-img}}
\end{figure}

  Опорный сигнал вместе с сигналом с генератора управляемого напряжением (ГУН) поступает в фазовый детектор. Фазовый детектор сравнивает частоты сигналов и на выходе получается сигнал следующего вида. Здесь синус называется характеристикой фазового детектора. Далее выходной сигнал фазового детектора поступает на вход фильтра нижних частот, который является линейным блоком и может быть описан передаточной функцией W(s). Выход фильтра поступает на вход генератора управляемого напряжением и происходит подстройка частоты ГУН. 
  
  Системы фазовой автоподстройки частоты широко применяются в различных системах, таких как телекоммуникационное оборудование, навигационное оборудование (GPS, ГЛОНАСС, Галилео). Также системы ФАПЧ применяются в компьютерах. Система фаза автоподстройки частоты нашли широкое применение и в военной индустрии. Например военными был разработан метод попеременной смены частоты сигнала для того чтобы в его было сложнее заглушить. Здесь также применяется система фазовой автоподстройки частоты.
    
Приведем систему дифференциальых уравнений описывающих ФАПЧ: 
    \begin{equation}\label{system}
 \begin{aligned}
 &\dot{x} = Ax + B(\operatorname{sin}(\theta_e) - \gamma)\text{,}\\[0.3pt]
 &\dot{\theta_e} = -K_{vco}C^T x -K_{vco}D(\operatorname{sin}(\theta_e) - \gamma))\text{,}
 \end{aligned}
\end{equation}
$A$~--- постоянная матрица $n \times n$, $B$ и $C$ постоянные $n$~--- мерные векторы, $D$~--- константа, $x(t)$~--- $n$-мерный вектор состояний системы, $K_{vco}$~--- коэффициент передачи.\vspace{-1mm}
 \begin{equation}\label{gamma}
 \begin{aligned}
\gamma = \frac{\omega_e^{free}}{K_{vco}\left(D-C^T A^{-1}B\right)}\text{,}
 \end{aligned}
\end{equation}
где $\omega^{free}_{e}=\omega_{ref}-\omega_{vco}^{free}$~--- разность частоты опорного сигнала и частоты свободных колебаний ГУН.
    
    Введём определение полосы захвата. Полоса захвата~--- максимальная разность по модулю частот опорного сигнала и ГУН $|\omega_p|$ такая, что система \eqref{system} глобально асимптотически устойчива для всех $0 \leqslant |\omega_e^{free}|<|\omega_p|$. 
    
Задача: оценить полосы захвата дла передаточных функций фильтров:
 \begin{align*}
&W(s) = \frac{1}{(1+\tau_{p1}s)(1+\tau_{p2}s)},\\[5pt]
&W(s) = \frac{(1+\tau_{z1}s)^2}{(1+\tau_{p1}s)^2},\\[5pt]
&W(s) = \frac{(1+\tau_{z1}x)(1+\tau_{z2}x)}{(1+\tau_{p1}x)(1+\tau_{p2}x)},\\
&0<\tau_{pi},\tau_{zj} < 1, \quad \tau_{pi} \neq \tau_{zj}, \quad i=1,2, \quad j=1,2.
 \end{align*}
    
В 1992 году Ленов~Г.\:А., Райтман~Ф., Смирнова~В.\:Б. доказали следующую теорему.
    
Введем обозначение
 \begin{equation}
 \begin{aligned}
\mid\nu\mid = \frac{0,5\pi\gamma}{\gamma \operatorname{arcsin} (\gamma) + \sqrt{1-\gamma^2}}
 \end{aligned}
\end{equation}
\begin{theorem}[Леонов~Г.\:А., Райтман~Ф., Смирнова В. Б.]
Пара $(A, B)$ вполне управляема, все собственные значения матрицы $A$ имеют отрицательные вещественные части и существуют числа $\varepsilon > 0$, $\delta > 0$, $\tau \geqslant 0$, и $\varkappa$, такие что имеют место неравенства:
 \begin{align}
&\operatorname{Re}\left( \varkappa W(ix)- \varepsilon\left[W(ix)\right]^2-\tau\left[ \overline{W(ix)}+ix \right]\left[W(ix)+ix \right]\right) \geqslant \delta \text{,}\enskip\forall x \in \mathbb{R} \label{first_th_eq}\\
&4\varepsilon\delta > (\varkappa\nu)^2
\end{align}
Тогда система \eqref{system} глобально асимптотически устойчива.
\end{theorem}

Оценим полосу захвата ФАПЧ для фильтра с передаточной функцией:
 \begin{equation}\label{filter1}
 \begin{aligned}
&W(s) = \frac{1}{(1+\tau_{p1}s)(1+\tau_{p2}s)} \text{,} \quad0 < \tau_{p1} < 1 \text{,} \quad 0 < \tau_{p2} <1
 \end{aligned}
\end{equation}
Подставим \eqref{filter1} в \eqref{first_th_eq} и перенесем все в левую часть неравенства. Тогда оценка $\nu$ будет наибольшей при следующих значениях параметров:
 \begin{equation*}
 \begin{aligned}
\varkappa = 1 \text{,} \enskip  \varepsilon = 1 - \tau - \delta \text{,} \enskip \tau = \tau_{p1}\tau_{p2} + \delta(\tau_{p1}^2+\tau_{p2}^2) \text{,} \enskip \delta = \frac{1-\tau_{p1}\tau_{p2}}{2(\tau_{p1}^2+\tau_{p2}^2 + 1)}
 \end{aligned}
\end{equation*}

Таким образом, получим следующую оценку:
\begin{equation}\label{filter1_max}
 \begin{aligned}
\nu^2 < \frac{(\tau_{p1}\tau_{p2} - 1)^2}{\tau_{p1}^2 + \tau_{p2}^2 + 1}
 \end{aligned}
\end{equation} 

%========================ФИЛЬТР 2=============================%

Оценим полосу захвата ФАПЧ для фильтра с передаточной функцией:
 \begin{equation}\label{filter2}
W(s) = \frac{(1+\tau_{z1}s)^2}{(1+\tau_{p1}s)^2}\text{,} \quad0 < \tau_{p1} < 1 \text{,} \quad 0 < \tau_{p2} <1 \text{,} \quad \tau_{p1} \neq \tau_{p2}
 \end{equation}
Подставим \eqref{filter2} в \eqref{first_th_eq} и перенесем все в левую часть неравенства. Положим $\varkappa = 1$, $\tau = 0$. Оценка $\nu$ будет наибольшей, если параметры $\varepsilon$, $\delta$ лежат на одной из граней выпуклого многоугольника, ограниченного прямыми $\delta = 0$, $\varepsilon = 0$ и 
 \begin{equation*}
\begin{aligned}
\varepsilon(\delta)=z^2 - z^4\delta \text{,} \quad \varepsilon(\delta)=q - z^2\delta \text{,}
\quad \varepsilon(\delta)=1 - \delta \text{,}
\end{aligned}
\end{equation*}
где $z = \frac{\tau_{p1}}{\tau_{z1}}$, $q = 2z - \frac{1}{2} - \frac{1}{2}z^2$. 
Тогда $4\varepsilon\delta$ определяется одним из следущих соотношений:

 \begin{equation}\label{filter2_max}
\begin{aligned}
\frac{q^2}{z^2}\text{,} \quad 1 \text{,} \quad \frac{4z^2}{1+z^2} \text{,} \quad \frac{4(1-q)(q-z^2)}{1-z^2} \text{,} \quad \frac{z^2-q}{z^2-1} - \left(\frac{z^2-q}{z^2-1}\right)^2
\end{aligned}
\end{equation}

Оценим полосу захвата ФАПЧ для фильтра с передаточной функцией:
 \begin{equation}\label{filter1}
 \begin{aligned}
&W(s) = \frac{1}{(1+\tau_{p1}s)(1+\tau_{p2}s)} \text{,} \quad0 < \tau_{p1} < 1 \text{,} \quad 0 < \tau_{p2} <1
 \end{aligned}
\end{equation}
Подставим \eqref{filter1} в \eqref{first_th_eq} и перенесем все в левую часть неравенства. Тогда оценка $\nu$ будет наибольшей при следующих значениях параметров:
 \begin{equation*}
 \begin{aligned}
\varkappa = 1 \text{,} \enskip  \varepsilon = 1 - \tau - \delta \text{,} \enskip \tau = \tau_{p1}\tau_{p2} + \delta(\tau_{p1}^2+\tau_{p2}^2) \text{,} \enskip \delta = \frac{1-\tau_{p1}\tau_{p2}}{2(\tau_{p1}^2+\tau_{p2}^2 + 1)}
 \end{aligned}
\end{equation*}

\begin{equation}\label{filter1_max}
 \begin{aligned}
\nu^2 < \frac{(\tau_{p1}\tau_{p2} - 1)^2}{\tau_{p1}^2 + \tau_{p2}^2 + 1}
 \end{aligned}
\end{equation} 

Оценим полосу захвата ФАПЧ для фильтра с передаточной функцией:\vspace{-1mm}
 \begin{equation}\label{filter3}
 \begin{aligned}
W(s) = \frac{1+\alpha_1\beta_1s + \alpha_2\beta_2s^2}{1+\alpha_1s + \alpha_2s^2} \text{,}\quad 0 < \beta_1 < \beta_2 < 1 \text{,}\quad 0 < \alpha_1 \text{,} \alpha_2
 \end{aligned}
\end{equation}
Подставим \eqref{filter3} в \eqref{first_th_eq} и перенесем все в левую часть неравенства. Для максимизации оценки $\nu$ положим: 

 \begin{equation}\label{filter3-params}
 \begin{aligned}
 \tau = 0 \text{,} \quad
 \varkappa = 1 \text{,} \quad
 \varepsilon = 1-\delta \text{,} \quad
 \delta = \frac{\alpha_1^2(1-\beta_1)\beta_1 - \alpha_2(1-\beta_2)}{\alpha_1^2(1-\beta_1^2) - 2\alpha_2(1-\beta_2)}
 \end{aligned}
\end{equation}
Чтобы гарантировать положительность $\delta$ потребуем:
 \begin{equation}\label{restriction-2}
 \begin{aligned}
\alpha_1^2 > \frac{\alpha_2(1-\beta_2)}{\beta_1(1-\beta_1)}
 \end{aligned}
\end{equation}
Тогда, подставив \eqref{filter3-params} в \eqref{first_th_eq} и применив \eqref{second_th_eq}, получим оценку:
 \begin{equation}
 \begin{aligned}
\nu^2 < 4\frac{[\alpha_1^2(1-\beta_1) - \alpha_2(1-\beta_2)][\alpha_1^2(1-\beta_1)\beta_1 - \alpha_2(1-\beta_2)]}{[\alpha_1^2(1-\beta_1^2) - 2\alpha_2(1-\beta_2)]^2}
 \end{aligned}
 \end{equation}
Эта оценка совпадает с оценкой, полученной в \cite{kuznetsov}.

для передаточной функции первого вида с ограничениями получим для начала оценку на new из условия теоремы оцените тканью будет максимальная если параметры положить следующим образом который проведены на слайде тогда оценка на нем квадрат получается следующим образом далее рассмотрим два графика слева проведена численная оценка не у квадрат полученная в матлабе при помощи функций of ming он а справа по график функции new квадрат построенной по аналитической оценки 7 далее из соотношения 8 можно получить максимальное омега евреи при котором система будет глобально синтетически устойчиво то есть из соотношения на ней и гамма мы можем найти гамма из соотношения на неё и омега и free мы можем найти amiga и free будет максимально таким же образом далее получим оценки на неё квадрат и соотношение 8 можно будет найти максимальный омега и free при котором систему под глобальность 10 устойчиво 

рассмотрим передаточную функцию 2 вида с ограничениями также на тау p 1 тау p 2 они должны быть еще не равны между собой можем положить не умаляя общности копы равна единице далее положим то у равна нулю тогда будет получаться что оценка на ней будет наибольшей если параметры опциона дельта лишаться одной из граней многоугольника ограниченного прямыми дельта равна нулю и epson равное нулю а также прямыми приведенными на слайде где z и q определяются следующими соотношения при тогда ни у квадрат не превзойдет одного из следующих соотношений конкретное соотношении 10 конкретный вид зависит непосредственно от z1 и патп-1 это уже т.д. далее рассмотрим график здесь синим цветом приведена численная оценка не у квадрат по первому втором условия теорема в лобби полученная при помощи функции f мин хён а красным цветом представлен график не у квадрат полученная по формулам 11 как максимум по всем граням многоугольника из графика видно что аналитическую оценку можно несколько улучшить поскольку график получены построены по 11 не жить несколько ниже чем график построен и попали 

далее рассмотрим передаточную функцию 3 вида функция проведено 12 при этом также можем положить не валяй общности каппа равна  динице положим то у равна нулю а также описано дельта следующим образом при этом условии теоремы дельта и обсе он должны быть строго положительным для этого достаточно потребовать положительность числителя дельта это также будет там гарантировать и знамя и положить на знаменателя дельта и при этом объём будет также положительно поэтому условия теоремы выполняются и получим оценку на new квадрат как говорилось ранее из оценки на ней квадрат можно получить уже оценку для omega a free и 

так что было проделано в работе для первого фильтра было найдено оценка полосы захвата для фильтра 2 вида было также найдено оценка полосы захвата которую можно улучшить фильтр 3 вида исследовался геннадий алексеем леоновым николай предел владимировичем кузнецовым а и для этого фильтра был восстановлен вывод оценки после захвата и оценка полученная в статье также совпало с оценкой выученное в данной работе всем большое спасибо за внимание


\end{document}