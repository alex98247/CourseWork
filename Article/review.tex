\documentclass[a4paper,article,14pt]{extarticle}
\usepackage[doublespacing]{setspace}
\usepackage{tocloft}

% Язык, кодировка, шрифт
\usepackage[utf8]{inputenc}
\usepackage[T2A]{fontenc}
\usepackage[english,russian]{babel}

\renewcommand\cftsecafterpnum{\vskip\baselineskip}
\renewcommand\cftsubsecafterpnum{\vskip\baselineskip}
\renewcommand\cftsubsubsecafterpnum{\vskip\baselineskip}

\begin{document}
\begin{singlespace}
  {\small
    \begin{center}
      \begin{minipage}{0.8\textwidth}
        \begin{center}
          {\normalsize \textbf{Рецензия  на статью}}\\[0.2cm]
          \textbf{Миронова А. В., Юлдашева Р. В., Юлдашева М. В., Кузнецова Н. В. <<Оценка полосы захвата для систем ФАПЧ
третьего порядка>>}
        
        \end{center}
      \end{minipage}
    \end{center}

   В данной статье авторами рассматривается проблема определения полосы захвата для систем фазовой автоподстройки частоты третьего порядка.

   Актуальность статьи не вызывает сомнения, так как исследование систем фазовой автоподстройки частоты является активно развивающейся областью современных исследований, что подтверждается большим количеством статей в международных высокорейтинговых журналах по математике и физике с высоким индексом цитирований. Из-за сложности исследования систем фазовой автоподстройки частоты третьего порядка основной акцент делается на численном моделировании. В данной работе авторами был предложен подход для получения аналитических оценок полос захвата систем ФАПЧ.

    Представленная на рецензию статья носит научный характер, материал изложен последовательно и логически корректно. Статья отвечает требованиям, предъявляемым к работам данного типа. Рекомендуется к опубликованию в открытой печати.

    \vspace{0.2cm}
    \noindent
      \begin{flushright}
      д.ф-м.н., проф. СПБГУ Т. Н. Мокаев
      \end{flushright}
  }
\end{singlespace}
\end{document}