\documentclass[a4paper]{article}
\usepackage[14pt]{extsizes} % для того чтобы задать нестандартный 14-ый размер шрифта
\usepackage[utf8]{inputenc}
\usepackage[russian]{babel}
\usepackage{setspace,amsmath}
\usepackage[left=20mm, top=15mm, right=15mm, bottom=15mm, nohead, footskip=10mm]{geometry} 
\usepackage[pdftex]{graphicx}
\usepackage{amsfonts}
\usepackage{listings}
\usepackage{amssymb}
\usepackage{color}
\usepackage{textcomp}
\definecolor{listinggray}{gray}{0.9}
\definecolor{lbcolor}{rgb}{0.9,0.9,0.9}
\usepackage{amsthm}% настройки полей документа
\lstset{
    language=java,
    upquote=true,
    aboveskip={1.5\baselineskip},
    columns=fullflexible,
    showstringspaces=false,
    extendedchars=true,
    breaklines=true,
    showtabs=false,
    showspaces=false,
    showstringspaces=false,
    identifierstyle=\ttfamily,
    keywordstyle=\color[rgb]{0,0,1},
    commentstyle=\color[rgb]{0.133,0.545,0.133},
    stringstyle=\color[rgb]{0.627,0.126,0.941},
}

\newtheorem{definition}{Определение}
\newtheorem{theorem}{Теорема}
 
\begin{document} % начало документа

\section{Введение}
Рассмотрим систему:
 \begin{equation}\label{etalon_system}
 \begin{aligned}
 &\dot{x} = Ax + b\upsilon_e(\theta_e)\\
 &\dot{\theta_e} = \omega_e^{free} - K_{vco}(c^*x + h\upsilon_e(\theta_e))
 \end{aligned}
\end{equation}
Найдем стационарные точки системы \eqref{etalon_system}
 \begin{equation}\label{transition}
 \begin{aligned}
 &x = -A^{-1}b\upsilon_e(\theta_e)\\
 &\upsilon_e(\theta_e) = \frac{\omega_e^{free}}{K_{vco}H(0)}
 \end{aligned}
\end{equation}
Возьмем $\theta_e = \theta_s$, для которых выполняется \eqref{transition}. Сделаем замену:
 \begin{equation}\label{replacement1}
 \begin{aligned}
 &z =x + A^{-1}b\upsilon_e(\theta_s)\\
 &\sigma = \theta_e 
 \end{aligned}
\end{equation}
После замены система \eqref{etalon_system} принимает вид:
 \begin{equation}
 \begin{aligned}
 &\dot{z} = Az + b(\upsilon_e(\sigma) - \frac{\omega_e^{free}}{K_{vco}H(0)})\\
 &\dot{\sigma} = -K_{vco}(c^*z + h(\upsilon_e(\sigma) - \frac{\omega_e^{free}}{K_{vco}H(0)}))
 \end{aligned}
\end{equation}

\section{Теорема}
Рассмотрим систему:
 \begin{equation}\label{system}
 \begin{aligned}
 &\dot{z} = Az + Bf(\sigma)\\
 &\dot{\sigma} = C^*z + Rf(\sigma)
 \end{aligned}
\end{equation}

Here A, B, C, and R are constant matrices of dimensions n × n, n × m, n × m, and n × n, respectively, whereas the components $\phi_k$, $1 \leq  k \leq m$, of the vector-valued function $f(\sigma)$ are scalar differen-tiable functions $\phi_k$ = $\phi_k(\sigma_k)$, where $\psi_k$ is the k-th component of the vector $\psi$.\\

Recall that any pendulum-like system with a single scalar non-linearity and an irreducible transfer function $\varkappa(p)$ can be written in the form \eqref{system} with m = 1 by a nonsingular linear transformation of phase coordinates. Further we assume that the functions $\phi_k(\sigma_k)$ satisfy the followingconditions: $\phi_k(\sigma_k) \equiv 0$\\
 \begin{equation}
 \begin{aligned}
&\mu_{1k} \leq \frac{d\varphi_k(\sigma_k)}{d\sigma_k} \leq \mu_{2k}, \forall \sigma_k \in \mathbb{R}\\
&\varphi_k(\sigma_k+\Delta_k) = \varphi_k(\sigma_k), \forall \sigma_k \in \mathbb{R}
 \end{aligned}
\end{equation}
Here the $\Delta_k$ are positive numbers. Sometimes, in the study of the specific pendulum-like systems, only one of the inequalities
 \begin{equation}
 \begin{aligned}
\mu_{1k} \leq \frac{d\varphi_k(\sigma_k)}{d\sigma_k}  or \frac{d\varphi_k(\sigma_k)}{d\sigma_k} \leq \mu_{2k}
 \end{aligned}
\end{equation}
is known, so we will assume the number $mu_{1k}$ to be either finite negative or $-\inf$, and the number $\mu_{2k}$ to be either finite positive or $\inf$.
When $\mu_{1k} = -\inf$ or $\mu_{2k} = \inf$, we will use the notation $\mu_{1k}^{-1} = 0$ or $\mu_{2k}^{-1} = 0$, respectively.\\

Let us introduce the numbers:
 \begin{equation}
 \begin{aligned}
\nu_k = \int_{0}^{\Delta_k} \varphi_k(\sigma) d\sigma (\int_{0}^{\Delta_k} \mid \varphi_k(\sigma) \mid d\sigma)^{-1}
 \end{aligned}
\end{equation}
the transfer matrix of system \eqref{system} from its “input” $f$ to its “output” $(-d\sigma/dt)$
 \begin{equation}
 \begin{aligned}
K(p) = C*(A - pI)^{-1}B - R
\end{aligned}
\end{equation}
and the diagonal m × m matrices:
 \begin{equation}
 \begin{aligned}
&\mu_1 = diag [\mu_{11}, . . . , \mu_{1m}],    \mu_2 = diag [\mu_{21}, . . . , \mu_{2m}],\\
&\nu = diag [\nu, _1. . . , \nu_m]
\end{aligned}
\end{equation}

\begin{theorem}
Suppose that the stationary set $\Lambda$ of system \eqref{system} consists of isolated points, the pair (A, B) is controllable, the matrix A is Hurwitz, and there exist diagonal m × m matrices $\varepsilon > 0, \delta > 0, \tau \geq 0$, and $\varkappa$ such that the following inequalities are valid:
 \begin{equation}
 \begin{aligned}
&Re(\varkappa K(ix)-K(ix)^*\varepsilon K(ix)-[K(ix)+\mu_1^{-1}ix]^*\tau[K(ix)+\mu_2^{-1}ix])\geq\delta, \forall x \in \mathbb{R}\\
&4\varepsilon\delta > (\varkappa\nu)^2
 \end{aligned}
\end{equation}
Then system \eqref{system} is gradient-like.
\end{theorem}

Очевидно, что систему \eqref{etalon_system} можно привести к системе  \eqref{system} положив: 
 \begin{equation}
 \begin{aligned}
&A=A\\
&B = b\\
&C = -K_{vco}c^*\\
&R = -K_{vco}h\\
&f(\sigma) = \upsilon_e(\sigma) - \frac{\omega_e^{free}}{K_{vco}H(0)}
\end{aligned}
\end{equation}
Положим далее:
 \begin{equation}
 \begin{aligned}
&\upsilon_e(\sigma) =  sin(\sigma)\\
&\frac{\omega_e^{free}}{K_{vco}H(0)} = \gamma
\end{aligned}
\end{equation}
\section{Теорема Декарта}
\begin{theorem}
Если многочлена записанного в стандартной форме действительные и все его корни также заведомо действительные, то число его положительных корней, если учитывать их кратности, равно числу перемен знаков в ряде его коэффициентов. Если же оно имеет и комплексные корни, то число это равно или на некоторое четное число меньше числа этих перемен знаков.
\end{theorem}
\section{Оценка области захвата для систем ФАПЧ с фильтром $\frac{1}{(1+\tau_{p1}x)(1+\tau_{p2}x)}$}
 Рассмотрим передаточную функцию:
 \begin{equation}\label{filter1}
 \begin{aligned}
K(x) = \frac{1}{(1+\tau_{p1}x)(1+\tau_{p2}x)} = \frac{1}{1+(\tau_{p1}+\tau_{p2})x + \tau_{p1}\tau_{p2}x^2}
 \end{aligned}
\end{equation}
Введем обозначения: $a = \tau_{p1}+\tau_{p2}$, $b = \tau_{p1}\tau_{p2}$\\
Рассмотрим первое условие теоремы:
\begin{equation}
 \begin{aligned}
Re(\varkappa K(ix)-K(ix)^*\varepsilon K(ix)-[K(ix)-ix]^*\tau[K(ix)+ix])\geq\delta
 \end{aligned}
\end{equation}
% $K(ix)=\frac{1}{1+aix + bix^2}=\frac{1-bx^2-iax}{(1-bx^2)^2 + a^2x^2}$\\\\
% $K(ix)^*K(ix)=\frac{1}{(1-bx^2)^2 + a^2x^2}$\\
% $\frac{\tau b^2x^6 + (\tau a^2-2*\tau*b)x^4 + (\varkappa*b+\tau)x^2 + (\varkappa-\varepsilon-\tau)}{(1-bx^2)^2 + a^2x^2}\geq\delta$\\
Подставим, рассматриваемую передаточную функцию в условие теоремы. В результате первое условие теоремы принимает следующий вид:
\begin{equation}\label{first_condition}
 \begin{aligned}
&\tau b^2t^3 + (\tau a^2-2 \tau b - \delta b^2)t^2 + (-\varkappa b+\tau-\delta a^2 + 2\delta b)t + (\varkappa-\varepsilon-\tau-\delta) \geq 0\\
&t \geq 0
 \end{aligned}
\end{equation}
Второе условие теоремы имеет вид: 
\begin{equation}
 \begin{aligned}
&4\varepsilon\delta > \nu^2\varkappa^2\\
&\frac{4\varepsilon\delta}{\varkappa^2} > \nu^2\\
 \end{aligned}
\end{equation}
Будем искать максимум $\frac{4\varepsilon\delta}{\varkappa^2}$

\subsection{Оценка максимума $\nu$ константой}
Очевидно, что для того чтобы выполнялось \eqref{first_condition} нужно  $\varkappa - \varepsilon - \tau - \delta \geq 0$
\begin{equation}\label{const_ineq}
 \begin{aligned}
&\varkappa \geq \varepsilon+\tau+\delta \\
&\varkappa^2 \geq \varepsilon^2 + \tau^2 + \delta^2 + 2\varepsilon\tau + 2\varepsilon\delta + 2\tau\delta\\
&2 -(\frac{2\varepsilon^2}{\varkappa^2} + \frac{2\delta^2}{\varkappa^2} + \frac{2\tau^2}{\varkappa^2} +\frac{4\varepsilon\tau}{\varkappa^2} + \frac{4\tau\delta}{\varkappa^2}) \geq \frac{4\varepsilon\delta}{\varkappa^2}\\
&2 > \frac{4\varepsilon\delta}{\varkappa^2} > \nu^2
 \end{aligned}
\end{equation}

\subsection{Точная оценка максимума $\nu$}
2. Так как $\varkappa - \varepsilon - \tau - \delta \geq 0$, то $ \varepsilon \leq \varkappa - \tau - \delta$. Для максимизации функции $\frac{4\varepsilon\delta}{\varkappa^2}$ возьмем $\varepsilon = \varkappa - \tau - \delta$\\
Тогда первое условие теоремы принимает вид:
\begin{equation}\label{ineq1} 
 \begin{aligned}
&\tau b^2t^2 + (\tau a^2-2 \tau b - \delta b^2)t + (-\varkappa b+\tau-\delta a^2 + 2\delta b) \geq 0\\
&t > 0
 \end{aligned}
\end{equation}
И будем искать максимум следующей функции: 
\begin{equation}
 \begin{aligned}
&f(z) = 4z-4z_1z - 4z^2\\
&z_1 = \frac{\tau}{\varkappa}, z = \frac{\delta}{\varkappa}
 \end{aligned}
\end{equation} 
Будем рассматривать $f(z)$, как функцию от $z$ с параметром $z_1$. Очевидно, что максимум этой функции достигается при $z_{max} = \frac{1-z_1}{2}$ и $f(z_{max}) = (1-z_1)^2$\\

Рассмотрим дискриминант уравнения \eqref{ineq1}:
\begin{equation}\label{discrimanant}
 \begin{aligned}
D = (\tau a^2-2 \tau b - \delta b^2)^2 - 4\tau b^2 (-\varkappa b+\tau-\delta a^2 + 2\delta b)
 \end{aligned}
\end{equation} 
Не умаляя общности, положим далее $\varkappa = 1$, тогда \eqref{discrimanant} примет вид:
\begin{equation}\label{discrimanant_transformed}
 \begin{aligned}
D = z_1^2\frac{1}{4}(b^4 + 8 b^3 - 4 a^2 b^2 - 16 a^2 b + 4 a^4 )+z_1\frac{1}{4}(- 2 b^4 + 8 b^3 + 4 a^2 b^2) + \frac{b^4}{4}
 \end{aligned}
\end{equation}
%$\frac{D}{\varkappa^2} = (\frac{\tau}{\varkappa} a^2-2 \frac{\tau}{\varkappa}  b - \frac{\delta}{\varkappa}  b^2)^2 - 4\frac{\tau}{\varkappa}  b^2 (b+\frac{\tau}{\varkappa} -\frac{\delta}{\varkappa}  a^2 + 2\frac{\delta}{\varkappa}  b) = (z_1 a^2-2 z_1  b - z  b^2)^2 - 4z_1  b^2 (b+z_1  - z  a^2 + 2z b) = (z_1 a^2-2 z_1  b - \frac{1-z_1}{2} b^2)^2 - 4z_1  b^2 (b+z_1  - \frac{1-z_1}{2}  a^2 + 1-z_1 b)=z_1^2(-4b^2-2a^2b^2+4b^3+(a^2-2b+\frac{b^2}{2})^2)-z_1(\frac{b^4}{2}+2b^3-a^2b^2+4b^2) + \frac{b^4}{4}$\\\\
Множество допустимых значений $z_1$ состоит из тех $z_1$ для которых:   $D \leq 0$ или $D \geq 0$ и $x_2 \leq 0$, где $x_2$ наибольший корень уравнения \eqref{ineq1}. Обозначим:
\begin{equation}
 \begin{aligned}
&A = \frac{1}{4}(b^4 + 8 b^3 - 4 a^2 b^2 - 16 a^2 b + 4 a^4)\\
&B = \frac{1}{4}(- 2 b^4 + 8 b^3 + 4 a^2 b^2)\\
&C = \frac{b^4}{4}
 \end{aligned}
\end{equation}
\subsubsection{$A  = 0$}
 При $b^4 + 8 b^3 - 4 a^2 b^2 - 16 a^2 b + 4 a^4  = 0$ \eqref{discrimanant_transformed} принимает вид:
\begin{equation}
 \begin{aligned}
D = z_1B + C
 \end{aligned}
\end{equation}
Предположим, $B \neq 0$, тогда очевидно, что корень уравнения $D = 0$ будет:
\begin{equation}
 \begin{aligned}
d = - \frac{1}{B}
 \end{aligned}
\end{equation}
При $B > 0$: $z_1 \in (0, d]$ $D \leq 0$, а при $z_1 \in [d, 1)$ $D \geq 0$\\
При $B < 0$: $z_1 \in [d, 1)$ $D \geq 0$, а при $z_1 \in [d, 1)$ $D \leq 0$\\
При $B = 0$: $z_1 \in (0, 1)$ $D \geq 0$
\subsubsection{$A \neq 0$}
Рассмотрим дискриминант уравнения $D = 0$:
\begin{equation}
 \begin{aligned}
D_1 = B^2 - 4AC
 \end{aligned}
\end{equation}
Заметим, что для исследуемых $\tau_{p1}$ и $\tau_{p2}$ $D_1 \geq 0$, тогда корни уравнения $D = 0$ имеют следующий вид:
\begin{equation}
 \begin{aligned}
d_{1,2} = \frac{-B\pm\sqrt{D_1}}{2A}
 \end{aligned}
\end{equation}
Не умаляя общности, будем считать: $0 < d_1 \leq d_2 <1$. \\
$A > 0$: Тогда при $z_1 \in (0, d_1] \cup [d_2, 1)$ $D \leq 0$, а при $z_1 \in [d_1, d_2]$ $D \geq 0$\\
$A < 0$: Тогда при $z_1 \in (0, d_1] \cup [d_2, 1)$ $D \geq 0$, а при $z_1 \in [d_1, d_2]$ $D \leq 0$

\subsubsection{Максимизация в зависимости от знака дискриминанта}
Рассмотрим, например, случай $A > 0$, остальные случаи рассматриваются аналогично. Как было показано выше, при $z_1 \in (0, d_1] \cup [d_2, 1)$ $D \leq 0$. Тогда имеем:
\begin{equation}
 \begin{aligned}
&M_1 = \max\limits_{-(z_1 a^2-2 z_1 b - \frac{1-z_1}{2} b^2)+\sqrt{D(z_1)}  \leq 0  }{|1-z_1|} = \max\limits_{-(z_1 a^2-2 z_1 b - \frac{1-z_1}{2} b^2)+\sqrt{D(z_1)}  = 0  }{|1-z_1|}\\
&z_1\in(0, d_1]\cup[d_2, 1)
 \end{aligned}
\end{equation} 
А при $z_1 \in [d_1, d_2]$ $D \geq 0$. Тогда почаем:
\begin{equation}
 \begin{aligned}
&M_2 = \max\limits_{z_1\in[d_1, d_2]}{|1-z_1|} = max\{|1-d_1|, |1-d_2|\}\\
&z_1\in[d_1, d_2]
 \end{aligned}
\end{equation}
Мы максимизировали функцию при отрицательном и неотрицательном дискриминанте. Положим максимум $M = \max \{M_1, M_2\}$. 

 
%\includegraphics[width=12cm]{Images/filter1.jpg}\\
%\includegraphics[width=20cm]{Images/filter1-1.jpg}

\section{Оценка области захвата для систем ФАПЧ с фильтром $\frac{(1+\tau_{z1}x)^2}{(1+\tau_{p1}x)^2}$}
 Рассмотрим передаточную функцию:
 \begin{equation}\label{filter2}
 \begin{aligned}
K(x) = \frac{(1+\tau_{z1}x)^2}{(1+\tau_{p1}x)^2}
 \end{aligned}
\end{equation}
Тогда первое условие теоремы принимает вид:
 \begin{equation}\label{second_condition}
 \begin{aligned}
&\tau_{p1}^4\tau t^3 +(- \tau_{z1}^4\varepsilon - \tau_{z1}^4\tau + 2\tau_{p1}^2\tau- \tau_{p1}^4\delta + \tau_{z1}^2\tau_{p1}^2\varkappa)t^2  +\\
&+( \tau- \tau_{z1}^2\varkappa - 2\tau_{z1}^2\varepsilon - \tau_{p1}^2\varkappa- 2\tau_{z1}^2\tau+ 4\tau_{z1}b\varkappa- 2\tau_{p1}^2\delta)t + (\varkappa-\varepsilon - \tau - \delta)  \geq 0\\
&t = x^2 \geq 0
 \end{aligned}
\end{equation}
Заметим, что при $\tau_{z1} \neq \tau_{p1}$ функция $\chi (x) = p^{-1}K(x)$ не приводима. Тогда по теореме пара $(A, B)$ управляема.
\subsection{$\tau = 0$}
Положим в \eqref{second_condition} $\tau = 0$, тогда первое условие теоремы принимает следующий вид:
 \begin{equation}\label{second_condition_tau_zero}
 \begin{aligned}
&(\tau_{z1}^2\tau_{p1}^2\varkappa - \tau_{z1}^4\varepsilon - \tau_{p1}^4\delta)t^2 +( 4\tau_{z1}\tau_{p1}\varkappa - \tau_{z1}^2\varkappa - 2\tau_{z1}^2\varepsilon - \tau_{p1}^2\varkappa - 2\tau_{p1}^2\delta)t + (\varkappa-\varepsilon - \delta)  \geq 0\\
&t = x^2 \geq 0
 \end{aligned}
\end{equation}
Будем искать максимум $\frac{4\varepsilon\delta}{\varkappa^2}$. Обозначим:
 \begin{equation}
 \begin{aligned}
&a = \tau_{z1}^2\tau_{p1}^2\varkappa - \tau_{z1}^4\varepsilon - \tau_{p1}^4\delta\\
&b = 4\tau_{z1}\tau_{p1}\varkappa - \tau_{z1}^2\varkappa - 2\tau_{z1}^2\varepsilon - \tau_{p1}^2\varkappa - 2\tau_{p1}^2\delta\\
&c = \varkappa-\varepsilon - \delta
 \end{aligned}
\end{equation}
Для того, что бы выполнялось \eqref{second_condition_tau_zero} нужно потребовать: $a > 0, c > 0$, тогда $b > 0$. По теореме Декарта уравнение не имеет положительных корней, т.е. выполняется \eqref{second_condition_tau_zero}.

\subsubsection{$\tau_{p1} > \tau_{z1}$}
Если предположить, что $\tau_{p1} > \tau_{z1}$, тогда из того, что $a > 0, c > 0$ следует, что $\delta < \varepsilon$. Положив $\delta = \varepsilon$ . Разделим на $\varepsilon > 0$ и не умаляя общности положим $\varepsilon = 1$ получим:
 \begin{equation}
 \begin{aligned}
&a = \tau_{z1}^2\tau_{p1}^2\varkappa - \tau_{z1}^4 - \tau_{p1}^4\\
&b = 4\tau_{z1}\tau_{p1}\varkappa - \tau_{z1}^2\varkappa - 2\tau_{z1}^2 - \tau_{p1}^2\varkappa - 2\tau_{p1}^2\\
&c = \varkappa-2
 \end{aligned}
\end{equation}
$$\lim_{\varkappa \to 2+} \frac{4\varepsilon\delta}{\varkappa^2} = 1$$
\subsubsection{$\tau_{p1} < \tau_{z1}$}
Аналогично предыдущему случаю, предположим, что $\tau_{p1} < \tau_{z1}$, тогда из того, что $a > 0, c > 0$ следует, что $\delta > \varepsilon$. Положив $\varepsilon = \delta$ . Разделим на $\delta > 0$ и не умаляя общности положим $\delta = 1$ получим:
 \begin{equation}
 \begin{aligned}
&a = \tau_{z1}^2\tau_{p1}^2\varkappa - \tau_{z1}^4 - \tau_{p1}^4\\
&b = 4\tau_{z1}\tau_{p1}\varkappa - \tau_{z1}^2\varkappa - 2\tau_{z1}^2 - \tau_{p1}^2\varkappa - 2\tau_{p1}^2\\
&c = \varkappa-2
 \end{aligned}
\end{equation}
$$\lim_{\varkappa \to 2+} \frac{4\varepsilon\delta}{\varkappa^2} = 1$$
%\includegraphics[width=12cm]{images/filter2.jpg}

\subsection{$\tau > 0$}
Предположим в \eqref{second_condition} $\tau > 0$ и разделим на $\tau_{p1}^4\tau$. Не умаляя общности, можем считать $\tau = 1$, тогда \eqref{second_condition} принимает следующий вид:
 \begin{equation}
 \begin{aligned}
&t^3 +at^2 +bt + c  \geq 0\\
&a = \frac{1}{\tau_{p1}^4}(- \tau_{z1}^4\varepsilon - \tau_{z1}^4 + 2\tau_{p1}^2- \tau_{p1}^4\delta + \tau_{z1}^2\tau_{p1}^2\varkappa)\\
&b = \frac{1}{\tau_{p1}^4}( 1- \tau_{z1}^2\varkappa - 2\tau_{z1}^2\varepsilon - \tau_{p1}^2\varkappa- 2\tau_{z1}^2+ 4\tau_{z1}\tau_{p1}\varkappa- 2\tau_{p1}^2\delta)\\
&c = \frac{1}{\tau_{p1}^4}(\varkappa-\varepsilon - 1 - \delta)\\
&t = x^2 \geq 0
 \end{aligned}
\end{equation}
Заметим, что по теореме Декарта: $a \geq 0, b \geq 0, c \geq 0$
\end{document}  % КОНЕЦ ДОКУМЕНТА !